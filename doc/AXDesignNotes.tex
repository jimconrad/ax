\documentclass[a4paper,11pt,twoside]{scrartcl}

\usepackage[utf8x]{inputenc}
\usepackage[ngerman]{babel}
\usepackage{longtable}
%\usepackage{fancyhdr}
\usepackage{graphicx}
\usepackage{tabularx}
\usepackage{amsmath}

\title{Arduino eXtension (AX) Services}
\author{Jim Conrad}

\begin{document}
\maketitle
\section{Introduction}
The Arduino eXtension (AX) services construct a multi-tasking operating system over the core Arduino ``language'' functions.
The target users for AX are advanced hobbyists and product developers who cannot use one of the legacy Real-Time Operating Systems for the Arduino.
AX APIs are packaged in an Arduino library and can be used with the Arduino IDE as well as the Eclipse CDT.
Even existing Arduino sketches can be adapted for use with AX.
\section{Design Decisions}
\subsection{IDE Compatibility}
\begin{enumerate}
\item AX is distributed as an Arduino Library, compatible with an advanced hobbyist's use of the Arduino IDE
\item AX applications can be developed using a professional IDE (e.g. the Eclipse CDT)
\end{enumerate}
\subsection{Board Compatibility}
\begin{enumerate}
\item AX leverages Arduino.h definitions to support multiple CPU architectures (e.g. 8-bit AVR, 32-bit ARM...)
\item Where possible, AX leverages existing Arduino software functions
\end{enumerate}

\end{document}